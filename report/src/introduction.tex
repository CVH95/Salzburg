\section{Introduction}
\label{sec:Introduction}

In robotics, obstacle avoidance in dynamic environments is a complex problem, that requires many different sub tasks to be coordinated in order to succeed. It extends for both mobile robots and manipulators. The robot has to be able to move without colliding with other bodies, which are also moving and, in most cases, in an unpredictable way. To do that, a complex sensor system is required, capable of detecting obstacles in real time and calculating its 3D real world position. In combination with it, the planner of the robot has to be able to process this information and calculate new routes on the go.

This project, carried out during the course of Robotics and Computer Vision 2, consisted of developing an Obstacle Avoidance Application for the UR5. The robot is set to move from an initial position Q\_start to a final configuration Q\_goal, like a pick and place scenario. Then, while the robot is working, a ball is placed somewhere between the two configurations, to intentionally obstruct the initial path planning. This ball is detected using a fixed stereo camera, which is also used to triangulate its location in the 3D space, relative to the robot, and finally replan the path so that the robot avoids the ball.

Within the project, we selected two focus areas: on the robotics side, we picked On-line Path Planning. Meanwhile, on the vision side, we chose 3D object tracking using Kalman Filter.

The solution was implemented in C++ with the help of some well-known software packages. We used the robotics framework ROS to integrate all the different sub-parts. Then, for vision we used OpenCV libraries and, in robotics, we used ROS C++ API as well as the RobWork and CAROS libraries, provided by SDU. 
\newpage
\subsection{Reading guide}

The following lines describe the structure of this report. The first part of it corresponds to the explanation of the theory concepts behind the system implemented, while the second part is more focused on describing the technical details of it and the experiments carried with their results at the Lab. 

Section \ref{sec:arch} presents a general overview of the system and its architecture. Section \ref{sec:calib} explains how the camera and the robot were calibrated. In Section \ref{sec:plan}, the fundamentals of the method used for achieving online-path planning in the dynamic environment are explained. For the vision part, object detection and triangulation are described in section \ref{sec:det} and \ref{sec:tri}, while how to track it with the Kalman filter is explained in section \ref{sec:kal}. This is the end of the theoretical part of the report. Regarding the next sections, the number \ref{sec:ros} explains in deep the ROS network created to implement all the previous algorithms. Sections 8 and 9 contain the experiments carried to evaluate the system and what we obtained and, finally, sections 10 and 11 present the discussion and conclusions over the whole project.


